\documentclass[12pt, letterpaper]{article}

\usepackage{graphicx}
\usepackage{amsmath}

\title{Experiment 3 Prelab}
\author{Jay Shen}
\date{January 2025}

\begin{document}

\maketitle

\section{Some basic visual observations}

At the beginning of the pinch-off, the drop forms a tapered droplet shape that tapers back out to the nozzle. 

As the pinch-off occurs, the droplet shape becomes significantly less smooth. The bottom of the droplet moves to form a spherical shape, and the neck becomes increasingly thin and tapers close to the drop. It is here that the neck achieves its smallest radius. The neck also tapers closer to the nozzle, but this effect is more pronounced near the droplet. 

As the pinch-off completes, the thinnest part of the neck, just above the droplet, disconnects. This effect does not seem to occur with linear speed, rather it speeds up as the neck gets thinner. The droplet then can free fall, and the neck recoils slightly from the disconnection. The neck then disconnects near the nozzle, and this end recoils too as the neck forms a drop of its own. 

After the neck droplet and the main droplet begin free falling, the recoil effects perturb their shapes in an oscillatory fashion, until they approach a spherical form.  

\section{Remarks about forces}

It is clear that gravity plays a big role here. Otherwise, the drop would not fall. 

The other force that distinguishes this process is surface tension. If there were no surface tension the water would fall continuously rather than discretely as drops. It holds the water in its cohesive shape at all stages except at the pinch-off in which it is overcome by gravity. 

The other factor we can identify is the viscosity. Comparing the glycerine and water videos, it seems glycerine forms a longer neck, which may reflect its different viscosity. 

\end{document}

\documentclass[12pt, letterpaper]{article}

\usepackage{graphicx}
\usepackage{amsmath}

\title{Experiment 4 Prelab}
\author{Jay Shen}
\date{February 2025}

\begin{document}

\maketitle

\begin{enumerate}
    \item {
        Since the photon is right polarized, $\Delta m_f = +1$. If the particle is in Zeeman state (c), then when excited it will end up in (B). 

        During decay $\Delta m_f = 0, \pm 1$, so from (C) the particle can end up in (a, b, c). 
    }
    \item{
        The photon can induce transitions with $\Delta m_f = \pm 1$ with equal probability. Then from (c), the excited state will be an equal superposition of (B, D): 
        \[
            \left|\psi\right\rangle = \frac{1}{\sqrt{2}}[\left|B\right\rangle + \left|D\right\rangle]
        \]
    }
    \item{
        Since the photon is left polarized, $\Delta m_f = -1$. If the particle is in Zeeman state (g), it cannot transition to a lower state, so it ends up in state (G). 
    }
    \item {
        $F = 3$
    }
    \item {
        $F = J + I$ so $I = F - J = 3 - \frac{1}{2} = \frac{5}{2}$
    }
\end{enumerate}

\end{document}
